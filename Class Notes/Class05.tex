\textbf{Ejemplo 2}

Sea $ \sigma \neq \varnothing $ y $ (a_n)_{n \in \mathbb{n}} $ una sucessión creciente, $A_n \contained A_{n+1} $
tenemos que:
\[ \liminf_{n \goesto \inf } A_n = \bigcup_{n \in \mathbb{N}} \bigcap_{m \geq n} A_m = \bigcup_{n \in \mathbb{n}} A_m = \bigcup_{n \in \mathbb{N}} A_n \]
y tambíen:
\[ \limsup_{n \goesto \infty} A_m = \bigcap_{n \in \mathbb{N}} \bigcap_{m \geq n} A_{m} = \bigcup_{m \geq n} A_m \]
\textbf{3.}

$ \sigma \neq \varnothing \quad (A_n)_{n \in \mathbb{N}} $ una sucessión decresiente, $A_n \supset A_{n+1} $
Tenemos que en general:

ver libro

\textbf{Cuarto Ejemplo}
Tenemos que $ \Omega = \set{0,1}^{\mathbb{N}} \quad A_n := \set{\omega \in \Omega | \omega_i = 0 \quad \forall_i \geq n+1 }$


sabemos que:

\[ \bigcap_{n \in \mathbb{N}} A_n = \set{(0,0,0, \ldots )} \]
\[ \liminf_{m \goesto \infty } A_n = \set{\omega \in \Omega | \exists \mathbb{N} \ldots} = \set{\omega \in \Omega | \sum_{i = n}^{\infty} \omega_i < \infty} \]

\textbf{5} ``Convergencia'' o `` Control de errores'' 
$\Omega \neq \varnothing$, $\epsilon > 0 $
\[ S(s_n)_{n \in \mathbb{N}}, \quad l,l_n: \Omega \goesto \RR \]
\[ A_{n,\epsilon} = \set{\omega \in \Omega | \abs{s_n(\omega) - s(\omega)} > \epsilon } \]

Estos fue una rte de analisis para la cual medimos la succesiones de funciones que son mayor que epsilon.

\[ \bigcap_{\epsilon > 0} \bigcap_{n \in \mathbb{N}} A_n^c = \set{\omega \notin \Omega | \forall n \in \mathbb{N} S_n(\omega) = s(\omega)} \]

\[ \bigcap_{\epsilon > 0 } \liminf_{n \goesto \infty} A_{n,\epsilon} \]
y esto es equivalente a:
\[ \set{\omega \in \Omega | \liminf_{n \goesto \infty} s_n(\omega) - s(\omega)} \]

\[ \bigcap_{\epsilon > 0} \limsup_{n \goesto \infty} A_{n,\epsilon}^c \]


\textbf{Sigmas Álgebras} 

$\Omega \neq \varnothing$.
\begin{itemize}
	\item $\varnothing \in A$,
	\item $A \in \mathfrak{A} \implies A^c  \in \mathfrak{A} $
	\item la union de dos elementos esta en $ \mathfrak{A} $ 
\end{itemize}


\begin{define}
	$(\sigma, \tau)$ espacio topologico, $B(\tau)$ se llaman Borel sigma ó sigma álgebra 
\end{define}

\begin{lema}
	$(\RR^d,\tau_d)$ es el espacio euclideano. Definimos $C^d := \qt{la familia de los conjuntos cerrados}$, despues $K^d := \qt{la familia de los conjuntos compactos}$, $J_a^d $ la familia de los intervalos abiertos
\end{lema}