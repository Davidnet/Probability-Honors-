Miremos la palabra CABEZA CUADRADA.
$n = 14$
\[ \binom{N} {K} \]
REvisar en las notas de distribución que existe un error dado por 
\[ \sigma = \set{(k_1, k_s) | k_i \in \set{0, \ldots m}, \sum_{i=1}^{r} K_i = m} \]

Para 3 tiros de un dado tenemos:
\[ p_1 = \frac{1}{3}, p_6 = \frac{1}{6} , p_2, p_3 = \ldots = p_5 = \frac{1}{9}\]
para el evento en que solo aparecen resultados pares tenemos: la probabilidad en que solo aparecen resultados pares es $ 0,059$

Por ultimo tema combinatorico, tenemos que $ m \in \mathbb{N}$,  $ M= \set{1, \ldots m} $ y definimos:
\[ V_{m,n} := \set{(b_1,\ldots , b_n) \in  M^n | b_i \leq b_{i+1} \]
y 
\[ U_{m,n} := \set{(a_1, \ldots, a_m) \in \set{0,1, \ldots, n}^m | \sum_{i=1}^{m} a_i = n} \]
Entonces, 
\[ #V_{m,n} = #U_{m,n} = \binom{m+n-1}{n} \]
tenemos un :
\[ A_b (j)  = \sum_{I=1}^{n} \mathbb{I}_{\set{j}} (b_i) \]
\[ (b_1, \ldots, b_n) = (1,1,1,1,2,2,2,3,3,3,3,3, \ldots , m-1) \]
la $\mathbb{I}$ es la funcion indicatriz.

\textbf{Ejemplo}
Elecciones, por ejemplo tenemos $m$ votantes y tenemos $n$ candidatos entonces cuantos resúltados distintos pueden ocurrir, y estos son:
\begin{align*}
\binom{m+n-1}{n}
\end{align*}
Observemos, que tenemos varias identidades:
\[ \frac{\binom{m + n}{n}}{\binom{m+n -1}{n}} = \frac{m+n +1}{n} \]
La formula anterior puede tener un error.
\textbf{Mecanica Estadistica}\\
Sistemas de particulas, cada particula esta en un estado eneregetico (estado micro), el estado macro es el estado de todo el ensemble:
\begin{itemize}
	\item $m$ estados, $(k_1, \ldots, k_m)$
\end{itemize}
Todas las particulas siguen el principio de exclusion de Pauli, que menciona: en cada estado mircro existe al máximo una particula.
Esta se llama la Estadistica de Fermi-Dirac.
Por lo cual tenemos que existen:
\[ \binom{n}{m} \]
estados macros.

De segundo tenemos la estadistica de Bose-Einstein: Que no respeta el principio de exclusion de Pauli, entonces eso explica que hay:
\[  \binom{n  +  m -1}{n} \]
Distribución uniforme sobre los estados macros es la distribución Bose-Einstein

Por ultimo tenemos \textbf{Estadisitica de Maxwell-Boltzteman}
Las particulas son distinguibles,
\begin{itemize}
	\item $m^n $ estados macros.
\end{itemize}

\section[2.4]{Espacios de Probabilidad Enumerables}
\begin{define}
	Sea $ \lambda = \varnothing $, $ \sigma = \set{x_1,x_2 \ldots} $ la cardinalidad inifinita de tal manera que hay una bijección con $ \mathbb{N} $:
	\begin{itemize}
		\item $ ( P_i) _{i \in \mathbb{N}}, \quad p_i \in [0,1], \quad \sum_{i=1}^{\infty} p_i = 1 $
		\item $\function{\mathbb{P}}{2^{\Omega}}{[0,1]} $
		
		\[ A \rightarrowtail \mathbb{P}(A)  \]
		\end{itemize}
	Se lllama distribución discreta de una moneda.
\end{define}

\textbf{Ejemplo} Tiro repetido e independiente de una moneda. $ \set{0,1} $, $\mathbb{P}((\set{\omega\in \set{0,1}^{\mathbb{N}}))$
	

Cual es la distribucion del tiempo aleatorio en $NN$ para la primer aparaicion de un resultado $1$

\begin{define}
	La distribucion de Possion $ p_\lambda, \lambda > 0 $ es una distibucion sobre $ \mathbb{N}, 2^{\mathbb{N}_0}  $ $P_\lambda(\set{i}) = e^{-\lambda} \lrp{\frac{\lambda^{i}}{i!}} $ 
	
\end{define}

\textbf{Torema limite de los cuentos raros}
Para $n \in \mathbb{N}$, $ p_n \frac{\lambda}{n}$, entonces
\[ \lim\limits_{n \rightarrow \infty} B_{n, pn} ( \set{U}) = P(\set{k}) \quad \forall \in \mathbb{N_0} \]

\textbf{La Paradoja de Banach-Tarski}
Para $  \RR^d $ y $d > 3$, $A, B \subseteq R^d$ y el interior de $A$ y $B$ son diferentes de vacio. Entonces existen particiones de $A$ y $B$ finitas tales que: $ A = A_1 \cup a_2 \ldots $ y $B = B_1 \cap B_2 \ldots$ de tal manera que $ A_i = B_i$

la preimagen es definida como:
\[ f^{-1}(B) = \set{\omega \in \sigma | f(\omega) \in B} \]
propiedades inmediatas son:
\begin{align*}
f^{-1}(\sigma) = \sigma \\
f^{-1}(\sigma') = \sigma
\end{align*}
\begin{lema}
	Sea $(b_j)_{j \in J} $ preimagen tenemos que respeta operaciones conjutistas bajo elementos no enumerables
\end{lema}

y con la imagen es estable bajo la union y tenemos que la imagen 